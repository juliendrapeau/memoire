\begin{comment}
\end{comment}

\chapter{Algorithmes variationnels quantiques}

%-----------------------------------------------------------------------------%

\begin{comment}
\subsection*{Plan}

\begin{enumerate}
    \item Décrire les algorithmes variationels en général
    \item Expliquer les objectifs de ces algorithmes
    \item Expliquer les avantages (exemple: algorithmes à court-terme, qubits bruités)
    \item Expliquer la chronologie avec les QAOA
    \item Expliquer comment est-ce qu'on peut utiliser ceux-ci comme générateur pour l'algorithme JVV.
    \item Pourquoi est-ce que QAOA est une approche non-locale?
\end{enumerate}
    
\subsection*{Références}

1. Cerezo, M. et al. Variational quantum algorithms. Nat Rev Phys 3, 625–644 (2021).

2. Bharti, K. et al. Noisy intermediate-scale quantum (NISQ) algorithms. Rev. Mod. Phys. 94, 015004 (2022).
\end{comment}



%-----------------------------------------------------------------------------%

\section{Algorithme quantique d'optimisation approximative}

\begin{comment}
\subsection*{Plan}
    
\begin{enumerate}
    \item Expliquer l'histoire et le lien avec le recuit quantique
    \item QAOA sees the whole graph?
\end{enumerate}

\subsection*{Références}

1. Farhi, E., Goldstone, J. and Gutmann, S. A Quantum Approximate Optimization Algorithm. Preprint at https://doi.org/10.48550/arXiv.1411.4028 (2014).

2. Kadowaki, T. and Nishimori, H. Quantum annealing in the transverse Ising model. Phys. Rev. E 58, 5355–5363 (1998).

3. Finnila, A. B., Gomez, M. A., Sebenik, C., Stenson, C. and Doll, J. D. Quantum annealing: A new method for minimizing multidimensional functions. Chemical Physics Letters 219, 343–348 (1994).

4. Farhi, E. et al. A Quantum Adiabatic Evolution Algorithm Applied to Random Instances of an NP-Complete Problem. Science 292, 472–475 (2001).

5. Farhi, E., Goldstone, J., Gutmann, S. and Sipser, M. Quantum Computation by Adiabatic Evolution. Preprint at https://doi.org/10.48550/arXiv.quant-ph/0001106 (2000).

6. Parler la adiabatic quantum computation https://arxiv.org/abs/1611.04471
\end{comment}

%-----------------------------------------------------------------------------%

\subsection{Approche adiabatique quantique}

\begin{comment}
\subsection{Plan}

\begin{enumerate}
    \item QAA est adiabatique alors que QAOA est contre-adiabatique.
\end{enumerate} 
\end{comment}

Le \textit{théorème adiabatique}, introduit par Born et Fock~\cite{bornBeweisAdiabatensatzes1928}, peut être énoncé simplement comme suit:

\begin{subtheorem}{Théorème adiabatique~\cite{bornBeweisAdiabatensatzes1928}}{theoreme-adiabatique}
    Un système physique demeure dans son état propre instantané si une perturbation donnée agit sur lui suffisament lentement et s'il y a un intervalle significatif entre la valeur propre et le reste du spectre de l'Hamiltonien.
\end{subtheorem}

Bien que différentes versions de ce théorème furent rigoureusement établies~\cite{albashAdiabaticQuantumComputing2018}, une version approximative de celui-ci, proposée par Messiah~\cite{messiahQuantumMechanics1999} et rectifiée par Amin~\cite{aminConsistencyAdiabaticTheorem2009}, est présentée ici dans l'objectif d'élucider les mécanismes du théorème. Un système quantique, décrit par un Hamiltonien dépendant du temps $H(t)$, évolue selon l'équation de Schrödinger

% , en commençant par Kant en 1950~\cite{katoAdiabaticTheoremQuantum1950},

\begin{align*}
   i \hbar \frac{\partial \ket{\psi(t)}}{\partial t} = H(t) \ket{\psi} \,.
\end{align*}

Considérons ici que l'Hamiltonien $H(t)$ peut s'écrire sous la forme $H(t) = \tilde{H}(s)$, où $s=t/T \in [0,1]$ est le temps adimensionnel, de manière à ce que $T$ contrôle le taux de variation dans le temps de $H(t)$. Soit $\ket{\varepsilon_{j} (s)}$ les états propres instantanés de $\tilde{H}(s)$ avec énergie $\varepsilon_{j}$ (potentiellement dégénérée) tel que

\begin{equation}
   \tilde{H}(s) \ket{\varepsilon_{j}(s)} = \varepsilon_{j}(s) \ket{\varepsilon_{j}(s)} \,,
\end{equation}

où $\varepsilon_{j}(s) < \varepsilon_{j+1}(s) \ \forall j,s$ et $j \in \set{ 0, 1, 2, \dots }$. L'approximation adiabatique indique qu'un état initial préparé dans un des états propres instantanés $\ket{\varepsilon_{j}(0)}$ demeure dans le même état propre instantané $\ket{\varepsilon_{j}(t)}$ à une phase globale près si

\begin{equation}
    T \gg \max_{s \in [0,1]} \frac{\lvert \braket{ \varepsilon_{i}(s) | \partial_{s} \tilde{H}(s) | \varepsilon_{j}(s) } \rvert }{\lvert \varepsilon_{i}(s) - \varepsilon_{j}(s) \rvert^{2} } \ \forall j \neq i
\end{equation}

et $\varepsilon_{i}(s) - \varepsilon_{j}(s) \neq  0$. \textcolor{mydarkred}{\textit{Rajouter une intuition et un lien avec le théorème.}}

L'approximation adiabatique est souvent utilisée à partir de l'état fondamental $\ket{\varepsilon_{0}(t)}$, menant à la définition du gap spectral entre l'état fondamental et le premier état excité du système $\Delta(s) = \varepsilon_{1}(s) - \varepsilon_{0}(s)$. Généralement, le maximum de $\braket{ \varepsilon_{i}(s) | \partial_{s} \tilde{H}(s) | \varepsilon_{j}(s)}$ est de l'ordre d'une valeur propre typique de $\tilde{H}$ et petit. Le minimum du carré de l'inverse du gap spectral $\Delta$ constitue alors un critère pratique pour quantifier le temps nécessaire à l'évolution adiabatique.

L'algorithme adiabatique quantique, introduit par Farhi~\cite{farhiQuantumComputationAdiabatic2000}, emploie un ordinateur quantique physique pour la résolution de problèmes d'optimisation combinatoire en se basant sur le théorème adiabatique quantique. Pour ce faire, le système physique est initialement préparé dans l'état fondamental d'un Hamiltonien de forçage $H_{D}$ facile à construire et dont l'état fondamental est simple à trouver. La solution du problème, encodée dans l'état fondamental de l'Hamiltonien du problème $H_{P}$ (voir la section~\ref{subsec:encodage-probleme}), est alors obtenue par une évolution adiabatique en transitionnant de l'Hamiltonien $H_{D}$ à l'Hamiltonien $H_{P}$. Plus précisément, l'Hamiltonien du système s'écrit comme


\begin{equation}
    \tilde{H}(s) = \left(1-s\right) H_{D} + s H_{P} \,.
\end{equation}

Ainsi, en assumant que le gap spectral entre l'état fondamental et l'état excité est non-nul, la solution du problème est toujours obtenue si l'évolution est suffisamment lente tel que guarantit par le théorème adiabatique quantique.

\textcolor{mydarkred}{\textit{Mais est-ce que le gap en non-nul en général?}}



%-----------------------------------------------------------------------------%

\subsection{Description de l'algorithme}
\label{subsec:description-algorithme}

\begin{comment}
subsection*{Plan}
    
\begin{enumerate}
    \item Décrire le \textit{Quantum Approximate Optimization Algorithm}
    \item Contre-diabacité
    \item Trottérisation
    \item 
\end{enumerate}

\subsection*{Références}

1. Farhi, E., Goldstone, J. and Gutmann, S. A Quantum Approximate Optimization Algorithm. Preprint at https://doi.org/10.48550/arXiv.1411.4028 (2014).
\end{comment}

Bien que le calcul adiabatique quantique est utilisé pour résoudre les problèmes d'optimisation combinatoire (\textcolor{mydarkred}{\textit{Source?}}), le temps nécessaire à l'évolution adiabatique constitue un facteur limitant pour de nombreux problèmes. L'algorithme quantique d'optimisation approximative (QAOA) propose ainsi une alternative, un raccourci à l'adiabacité: la contre-diabacité.

\textcolor{mydarkred}{\textit{Expliquer la contre-diabacité.}}



\begin{equation}
    \label{eq:final-state}
    \ket{\psi(\vec{\beta}, \vec{\gamma})} = \underbrace{U_D(\beta_p) U_P(\gamma_p) \cdots U_D(\beta_1) U_P(\gamma_1)}_{p \text { fois }} \ket{\psi_{0}}
\end{equation}

%-----------------------------------------------------------------------------%

\subsection{Initialisation des paramètres}
\label{subsec:initialisation-parametres}

\subsection*{Plan}

\begin{enumerate}
    \item Décrire l'importance d'une bonne initialisation des paramètres (\textit{barren plateau}, non-convexité des paramètres)
    \item Énumérer les principales méthodes
    \item Expliquer l'initialisation aléatoire par grille
    \item Expliquer \textit{TQA}
\end{enumerate}

\subsection*{Références}

1. Bittel, L. and Kliesch, M. Training Variational Quantum Algorithms Is NP-Hard. Phys. Rev. Lett. 127, 120502 (2021).

2. Anschuetz, E. R. and Kiani, B. T. Beyond Barren Plateaus: Quantum Variational Algorithms Are Swamped With Traps. Nat Commun 13, 7760 (2022).

3. Akshay, V., Philathong, H., Morales, M. E. S. and Biamonte, J. D. Reachability Deficits in Quantum Approximate Optimization. Phys. Rev. Lett. 124, 090504 (2020).
    
4. Cain, M., Farhi, E., Gutmann, S., Ranard, D. and Tang, E. The QAOA gets stuck starting from a good classical string. Preprint at https://doi.org/10.48550/arXiv.2207.05089 (2022).

Peut-être une référence de plus qui traite directement des barrens plateau?

%-----------------------------------------------------------------------------%

\subsection{Encodage du problème}
\label{subsec:encodage-probleme}

\begin{comment}
\subsection*{Plan}

\begin{enumerate}
    \item Introduire la fonction de coût
    \item Introduire le modèle d'Ising et le modèle QUBO
    \item Décrire la transformation d'Ising pour NAE3SAT et 1in3SAT
    \item Prouver la transformation d'Ising pour NAE3SAT et 1in3SAT
\end{enumerate}

\subsection*{Références}

1. Lucas, A. Ising formulations of many NP problems. Frontiers in Physics 2, (2014).
\end{comment}

\begin{figure}[h]
    \centering
    \includegraphics[width=0.5\textwidth]{figures/ising-mapping}
    \caption{}
    \label{fig:...}
\end{figure}


%-----------------------------------------------------------------------------%

\subsection{Choix du forçage}

\subsection*{Plan}

\begin{enumerate}
    \item Expliquer le but du forçage
    \item Expliquer forçage en X
    \item Expliquer le forçage de Grover
    \item Énumérer les forçages populaires
    \item Est-ce que le mixer doit être un vecteur propre de l'état initial?
\end{enumerate}

\subsection*{Références}

%-----------------------------------------------------------------------------%

\section{Approche quantique des opérateurs alternants avec forçage de Grover}

\begin{comment}
\subsection*{Plan}

\begin{enumerate}
    \item Décrire le \textit{Quantum Alternating Operator Ansatz}
    \item Décrire \textit{Grover-Mixer Quantum Alternating Operator Ansatz}
\end{enumerate}

\subsection*{Références}

1. Hadfield, S. et al. From the Quantum Approximate Optimization Algorithm to a Quantum Alternating Operator Ansatz. Algorithms 12, 34 (2019).

2. Bärtschi, A. and Eidenbenz, S. Grover Mixers for QAOA: Shifting Complexity from Mixer Design to State Preparation. in 2020 IEEE International Conference on Quantum Computing and Engineering (QCE) 72–82 (2020). doi:10.1109/QCE49297.2020.00020.
\end{comment}

\begin{equation}
    U_D^{\mathrm{Grover}} = U_{S}\left[ \mathds{1} -\left(1-e^{-i \beta}\right) (\ket{0}\!\bra{0})^{\otimes n} \right] U_{S}^{\dagger} \,,
\end{equation}

%-----------------------------------------------------------------------------%

\section{Échantillonage et biais}

\subsection*{Plan}

\begin{enumerate}
    \item Expliquer l'importance de l'échantillonnage non-biaisé
    \item Expliquer le problème d'échantillonage associé au recuit quantique
    \item Expliquer le problème d'échantillonage associé à QAOA
    \item Expliquer pourquoi GM-QAOA résout ce problème (ne pas oublier d'expliquer les inconvénients de cette méthode)
\end{enumerate}

\subsection*{Références}

1. Zhang, Z. et al. Grover-QAOA for 3-SAT: Quadratic Speedup, Fair-Sampling, and Parameter Clustering. Preprint at https://doi.org/10.48550/arXiv.2402.02585 (2024).

2. Mandrà, S., Zhu, Z. and Katzgraber, H. G. Exponentially Biased Ground-State Sampling of Quantum Annealing Machines with Transverse-Field Driving Hamiltonians. Phys. Rev. Lett. 118, 070502 (2017).

3. Matsuda, Y., Nishimori, H. and Katzgraber, H. G. Ground-state statistics from annealing algorithms: quantum versus classical approaches. New J. Phys. 11, 073021 (2009).

Plus de sources sur le fair sampling pour QAOA?
