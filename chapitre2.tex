\begin{comment}
\end{comment}

\chapter{Échantillonnage quasi aléatoire et comptage approximatif}

\subsection*{Plan}

\begin{enumerate}
    \item Énoncer brièvement l'algorithme de JVV pour introduire la section
    \item Re-mentionner l'importance du comptage approximatif (en autres en comparaison avec le comptage exact)
    \item Mentionner les concepts nécessaires pour l'algorithme de JVV (auto-réductibilité, échantillonnage quasi aléatoire, comptage approximatif)
\end{enumerate}
\subsection*{Références}

%-----------------------------------------------------------------------------%

\section{Auto-réductibilité}

\subsection*{Plan}

\begin{enumerate}
    \item Introduire les concepts d'auto-réductibilité
\end{enumerate}

\subsection*{Références}

1. Hemaspaandra, L. A. The Power of Self-Reducibility: Selectivity, Information, and Approximation. Preprint at https://doi.org/10.48550/arXiv.1902.08299 (2019).

\subsection*{Brouillon}

Introduction de "Autoreducibility" par Trakhtenbrot.

Introduction de "Self-reducibility" par Schnorr et Meyer/Paterson.

Survey paper de Balcázar, Selke, Allender.

Explication simple par Hemaspaandra.

%-----------------------------------------------------------------------------%

\section{Échantillonnage quasi aléatoire}

\subsection*{Plan}

\begin{enumerate}
    \item Introduire les FPAUS
    \item Introduire la distance en variation totale et la non-uniformité
\end{enumerate}

\subsection*{Références}

%-----------------------------------------------------------------------------%

\section{Comptage approximatif}

\subsection*{Plan}

\begin{enumerate}
    \item Introduire les FPRAS
    \item Introduire les algorithmes de comptage classique connus (ex.: Stockmeyer et JVV)
\end{enumerate}

\subsection*{Références}

1. Stockmeyer, L. The complexity of approximate counting. in Proceedings of the fifteenth annual ACM symposium on Theory of computing 118–126 (Association for Computing Machinery, New York, NY, USA, 1983). doi:10.1145/800061.808740.

%-----------------------------------------------------------------------------%

\section{Algorithme de Jerrum-Valiant-Vazirani}

\subsection*{Plan}

\begin{enumerate}
    \item Introduire le but de l'algorithme de JVV
    \item Vulgariser l'algorithme de JVV
    \item Introduire rigoureusement l'algorithme de JVV
\end{enumerate}

\subsection*{Références}

1. Jerrum, M. R., Valiant, L. G. and Vazirani, V. V. Random generation of combinatorial structures from a uniform distribution. Theoretical Computer Science 43, 169–188 (1986).
2. Huber, M. Exact sampling and approximate counting techniques. in Proceedings of the thirtieth annual ACM symposium on Theory of computing 31–40 (Association for Computing Machinery, New York, NY, USA, 1998). doi:10.1145/276698.276709.

