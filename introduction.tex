\Introduction   % Chapitre qui ne sera pas numéroté si IntroConcluSansNombre est Vrai

%-----------------------------------------------------------------------------%

En tirant parti du parallélisme, de la superposition et de l'intrication, les algorithmes quantiques remettent en question les limites classiques du calcul, promettant des accélérations exponentielles pour certains types de problèmes. L'algorithme de Shor permet par exemple de factoriser des entiers naturels en offrant une accélération superpolynomiale par rapport aux 
meilleurs algorithmes classiques connus~\cite{shorAlgorithmsQuantumComputation1994}. L'utilité actuelle du calcul quantique demeure toutefois incertaine pour plusieurs raisons. Les algorithmes quantiques ne s'appliquent présentement qu'à une classe de problèmes restreints, possédant un éventail d'applications limitées. De plus, les ordinateurs quantiques du moment, nommés ordinateurs quantiques à échelle intermédiaire bruités, sont affligés par différentes complications rendant tout calcul excessivement difficile: un nombre limité de qubits, une connectivité limitée et la présence d'erreurs limitant la taille du circuit.

Pour contourner ces difficultés, les algorithmes variationnels quantiques (« Variational Quantum Algorithms ») (VQA) fût conçus pour exploiter les mécanismes de l'algorithmie quantique tout en prenant avantage de la puissance du calcul classique~\cite{cerezoVariationalQuantumAlgorithms2021}. Ceux-ci reposent sur des circuits quantiques paramétrés dont les paramètres sont optimisé par un optimiseur classique pour effectuer des tâches computationnelles complexes. Étant basée sur l'optimisation de paramètres, cette stratégie est en mesure de considérer toutes les difficultés précédentes d'un coup tout en limitant la taille des circuits quantiques. De nombreux travaux au cours des dernières années ont tenté d'améliorer les VQA en introduisant des stratégies pour mieux exploiter la structure des problèmes. Malgré ces études, il est présentement inconnu si ces algorithmes possèdent un avantage par rapport aux algorithmes classiques. 

Les VQA sont typiquement appliqués à des problèmes d'optimisation combinatoire, comme le problème de coupe maximale et d'ensembles indépendants. Comme des solveurs classiques pour ce type de problème ont été développés au fil de décennies, il est peu évident de montrer la supériorité des VQA. Le projet de ce mémoire offre une perspective différente sur cette approche. Plutôt que d'entrer en compétition avec des algorithmes classiques déjà performants, les algorithmes variationnels quantique sont appliqués à la résolution de problèmes vastement plus complexes: les problèmes de comptage. Ceux-ci, éléments de la classe complexité \textsf{\#P}, se place plus haut que les problèmes d'optimisation dans la hiérarchie de la complexité computationnelle. 

Pour ce faire, l'équivalence entre le comptage approximatif et l'échantillonnage quasi uniforme, montré par Jerrum, Valiant et Vazirani~\cite{jerrumRandomGenerationCombinatorial1986}, est utilisée. Cette correspondance donne lieu un algorithme randomisé de comptage approximatif, surnommé algorithme de JVV, permettant l'estimation à un facteur multiplicatif près du nombre de solutions à un problème de comptage auto-réduction par l'échantillonnage d'une distribution aléatoire quasi uniforme. Cet algorithme se base sur la propriété d'auto-réductibilité de certains problèmes, c'est-
à-dire la possibilité d'utiliser la structure inhérente au problème pour résoudre celui-ci. La distribution préparée par un VQA peut alors être utilisée comme pour la génération de solutions à cet algorithme, résolvant de cette façon le problème de comptage. 

L'algorithme VQCount, introduit dans ce mémoire, permet de faire le pont entre les VQA et les problèmes de comptage. VQCount prépare une distribution d'états contenant une superposition de solutions au problème de comptage avec une haute probabilité en utilisant l'ansatz quantique à opérateurs alternés avec forçage de Grover~\cite{bartschiGroverMixersQAOA2020}. L'algorithme de comptage approximatif est alors employé pour obtenir un estimé du nombre de solutions. Toutefois, pour prendre avantage de la propriété d'auto-réductibilité, une procédure d'auto-réduction, modifiant le circuit quantique, est utilisée dans l'algorithme VQCount. 

Grâce à des simulations de réseaux de tenseurs, cette étude évalue la performance de VQCount sur des circuits quantiques de faible profondeur appliqués à des instances synthétiques de deux problèmes \textsf{\#P}-difficiles: positif \#1-in-3SAT et positif \#NAE3SAT. Les solutions sont générées à l'aide de l'algorithme quantique d'optimisation approximative (QAOA)~\cite{farhiQuantumApproximateOptimization2014} et de l'ansatz quantique à opérateurs alternants avec forçage de Grover (GM-QAOA). L'analyse met en évidence un équilibre entre la probabilité de succès et l'uniformité de l'échantillonnage, qui permet d'obtenir un gain exponentiel en efficacité par rapport à une approche naïve par rejet. Ces résultats révèlent à la fois les atouts et les limites des algorithmes variationnels quantiques pour le comptage approximatif.

% \textcolor{mydarkred}{\textit{CHATGPT!}}

Le chapitre~\ref{cha:complexite-du-comptage} décrit en profondeur les problèmes de comptage avec le langage de la théorie de la complexité. L'algorithme de JVV et ses concepts sous-jacents, c'est-à-dire l'auto-réductibilité, l'échantillonnage aléatoire et le comptage approximatif, sont présentés au chapitre~\ref{cha:echantillonnage-quasi-uniforme-comptage-approximatif-randomise}. Les algorithmes variationnels quantiques sont présentés de manière générale au chapitre~\ref{cha:algorithmes-variationnels-quantiques} avec un focus sur QAOA et GM-QAOA. L'algorithme VQCount, résultat de ce travail, est détaillé au chapitre~\ref{cha:comptage-variationnel-quantique}. La performance de l'algorithme VQCount est finalement caractérisé au chapitre~\ref{cha:resolution-de-problemes-avec-vqcount}. L'annexe~\ref{ann:auto-reductibilite-du-circuit-gm-qaoa} montre que la procédure d'auto-réduction est bien applicable au, alors que l'annexe~\ref{ann:simulation-circuits-quantiques-avec-reseaux-de-tenseurs} introduit les méthodes de réseaux de tenseurs utilisé pour la simulation de circuit quantique de l'algorithme VQCount.

Le travail effectué au cours de ce projet a mené à la publication d'un manuscrit sur \textit{arXiv}: \textcolor{mydarkred}{\textit{Rajouter le lien}}. Ce mémoire prend ainsi son inspiration de cet écrit.