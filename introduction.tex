\Introduction   % Chapitre qui ne sera pas numéroté si IntroConcluSansNombre est Vrai

%-----------------------------------------------------------------------------%

En exploitant les principes du parallélisme, de la superposition et de l'intrication, les algorithmes quantiques remettent en question les limites classiques du calcul, promettant des accélérations exponentielles pour certains types de problèmes. L'algorithme de Shor, par exemple, permet de factoriser des entiers naturels en offrant une accélération superpolynomiale par rapport aux 
meilleurs algorithmes classiques connus~\cite{shorAlgorithmsQuantumComputation1994}. Bien que le calcul quantique soit prometteur, son utilité pratique reste floue pour diverses raisons. Les algorithmes quantiques ne s'appliquent en ce moment qu'à une classe de problèmes restreints, et leurs applications restent limitées. En outre, les ordinateurs quantiques actuels, appelés ordinateurs quantiques à échelle intermédiaire bruités, sont affligés par différentes complications rendant tout calcul excessivement difficile: un nombre restreint de qubits, une connectivité limitée et la présence d'erreurs entravant la taille des circuits.

Pour contourner ces difficultés, les algorithmes variationnels quantiques (« Variational Quantum Algorithms ») (VQA) furent conçus pour exploiter les mécanismes du calcul quantique, tout en tirant profit de la puissance du calcul classique~\cite{cerezoVariationalQuantumAlgorithms2021}. Ces algorithmes visent à résoudre des problèmes complexes en utilisant l'état préparé par un circuit quantique paramétré, dont les paramètres sont ajustés par un optimiseur classique pour minimiser la fonction de coût du problème. Étant basée sur l'optimisation de paramètres, cette stratégie permet de considérer toutes les difficultés précédentes d'un seul coup, tout en limitant la taille des circuits quantiques. Au cours des dernières années, de nombreux travaux ont tenté de perfectionner les VQA en développant par exemple des techniques permettant une meilleure exploitation de la structure des problèmes. Toutefois, il reste encore à déterminer si ces algorithmes présentent un réel avantage par rapport aux algorithmes classiques.  

Les VQA sont principalement utilisés pour résoudre des problèmes d'optimisation combinatoire, tels que le problème de coupe maximale et le problème des ensembles indépendants. Comme des solveurs classiques pour ce type de problème ont été développés au fil des décennies, il est difficile de démontrer la supériorité des VQA. Le projet de ce mémoire propose une approche différente pour aborder cette question. Plutôt que de rivaliser avec des algorithmes classiques hautement performants, les VQA sont appliqués à la résolution de problèmes vastement plus complexes: les problèmes de dénombrement, ou plus simplement les problèmes de comptage. Au lieu de chercher une solution quasi optimale, ce type de problème s'intéresse plutôt à déterminer le nombre de solutions. Les problèmes de comptage, éléments de la classe complexité \textsf{\#P}, se placent au-dessus des problèmes d'optimisation dans la hiérarchie de la complexité computationnelle.

Pour ce faire, l'équivalence entre le comptage approximatif et l'échantillonnage quasi uniforme, établie par Jerrum, Valiant et Vazirani~\cite{jerrumRandomGenerationCombinatorial1986}, est utilisée. Cette correspondance donne lieu à un algorithme randomisé de comptage approximatif, surnommé algorithme de JVV, permettant l'estimation à un facteur multiplicatif près du nombre de solutions à un problème de comptage auto-réductible si les solutions peuvent être échantillonnées suffisamment uniformément. Cet algorithme exploite la propriété d'auto-réductibilité de certains problèmes, c'est-à-dire la capacité à utiliser la structure inhérente au problème pour sa résolution.

L'algorithme VQCount, introduit dans ce mémoire, fait le pont entre les VQA et les problèmes de comptage. Cet algorithme montre qu'il est possible d'utiliser les VQA pour construire un solveur approximatif aux problèmes \textsf{\#P}. Pour ce faire, VQCount prépare une distribution d'états contenant une superposition de solutions au problème de comptage avec une haute probabilité en utilisant l'ansatz quantique à opérateurs alternés (QAOA)~\cite{hadfieldQuantumApproximateOptimization2019}. Après avoir modifié le circuit quantique de QAOA pour tirer avantage de la propriété d'auto-réductibilité, VQCount emploie l'algorithme de comptage approximatif pour estimer le nombre de solutions à partir de la distribution préparée. Si QAOA échantillonne des solutions suffisamment près de l'uniformité avec un taux de succès fini, alors seulement un nombre polynomial d'échantillons est nécessaire pour estimer le nombre de solutions avec l'algorithme de JVV. 

Grâce aux simulations de réseaux de tenseurs, cette étude évalue la performance de VQCount sur des circuits quantiques de faible profondeur appliqués à des instances synthétiques de deux problèmes \textsf{\#P}-difficile: positif \#1-in-3SAT et positif \#NAE3SAT. Les solutions sont produites à l'aide de l'algorithme quantique d'optimisation approximative (QAOA)~\cite{farhiQuantumApproximateOptimization2014} et de l'ansatz quantique à opérateurs alternants avec forçage de Grover (GM-QAOA)~\cite{bartschiGroverMixersQAOA2020}. Cette analyse met en évidence un équilibre entre la probabilité de succès et l'uniformité de l'échantillonnage, qui permet d'obtenir un gain exponentiel en efficacité par rapport à une approche naïve par rejet. Ces résultats révèlent à la fois les atouts et les limites des algorithmes variationnels quantiques pour le comptage approximatif.

Le chapitre~\ref{cha:complexite-du-dénombrement} explore en détails les problèmes de dénombrement avec le langage de la théorie de la complexité. L'algorithme de JVV et ses concepts sous-jacents, c'est-à-dire l'auto-réductibilité, l'échantillonnage aléatoire et le comptage approximatif, sont présentés au chapitre~\ref{cha:echantillonnage-quasi-uniforme-comptage-approximatif-randomise}. Les algorithmes variationnels quantiques sont présentés de manière générale au chapitre~\ref{cha:algorithmes-variationnels-quantiques} avec un accent sur QAOA et GM-QAOA. L'algorithme VQCount, fruit de ce travail, est détaillé au chapitre~\ref{cha:comptage-variationnel-quantique}. La performance de l'algorithme VQCount est finalement caractérisée dans le chapitre~\ref{cha:resolution-de-problemes-avec-vqcount}. L'annexe~\ref{ann:auto-reductibilite-du-circuit-gm-qaoa} démontre une expression fermée de l'état préparé par un circuit GM-QAOA, alors que l'annexe~\ref{ann:simulation-circuits-quantiques-avec-reseaux-de-tenseurs} introduit les méthodes de réseaux de tenseurs utilisées pour la simulation de circuit quantique de l'algorithme VQCount.

Le travail effectué au cours de ce projet a mené à la publication d'un manuscrit sur arXiv: \textcolor{mydarkred}{\textit{Rajouter le lien}}. Ce mémoire prend ainsi son inspiration de cet écrit.