\begin{comment}
\end{comment}


\chapter{Expression fermée de l'état du circuit GM-QAOA}
\label{ann:expression-fermee-etat-circuit-gm-qaoa}

%-----------------------------------------------------------------------------%

\begin{maintheorem}{Expression fermée de l'état du circuit GM-QAOA}{expression-fermee-etat-gm-qaoa}
    Pour un espace de solutions non contraint, l'état préparé par un circuit GM-QAOA $\ket{\psi(\vec{\gamma}, \vec{\beta})}$ de profondeur $p$ pour $n$ qubits s'exprime sous la forme de l'expression récursive fermée suivante:

    \begin{equation}
        \label{eq:recursive-formula}
        \ket{\psi} = \frac{1}{\sqrt{ N }} \left( \sqrt{ \lvert G \rvert } F_{p} \ket{g} + \sum_{k} \sqrt{ \lvert E^{(k)} \rvert } \bar{F}^{(k)}_{p} \ket{e^{(k)}} \right) \,,
    \end{equation}
    où $N=2^{n}$, $F_{0}=\bar{F}_{0}^{(k)}=1$ et
    \begin{align}
        F_{p} &= F_{p-1} - \frac{1}{N} (1-e^{-i\beta_{p}}) \left( \lvert G \rvert   F_{p-1} + \sum_{k'} \lvert E^{(k')} \rvert \bar{F}^{(k')}_{p-1} e^{-i\gamma_{p}\varepsilon_{k'}} \right) \,, \\
        \bar{F}^{(k)}_{p} &= e^{-i\gamma_{p} \varepsilon_{k}}\bar{F}_{p-1}^{(k)} - \frac{1}{N} (1-e^{-i\beta_{p}}) \left( \lvert G \rvert   F_{p-1} + \sum_{k'} \lvert E^{(k')} \rvert \bar{F}^{(k')}_{p-1} e^{-i\gamma_{p}\varepsilon_{k'}} \right) \,,
    \end{align}
    pour les états fondamentaux $\ket{g} \in G$ et les états excités $\ket{e^{(k)}} \in E^{(k)}$ de niveau $k$.
\end{maintheorem}

\begin{proof}
    
Soit $\varphi$ l'instance d'un problème de \textsf{\#P} de taille $n$ que nous voulons résoudre avec GM-QAOA. Cette instance peut être encodée dans un hamiltonien de problème $H_{P}$ avec des valeurs propres $g = 0$ pour les solutions et $e^{(k)} = \varepsilon_{k} \in \mathbb{R}_{>0}$ pour les non-solutions. L'opérateur $U_{P}$ correspondant avec $H_{P}$ s'écrit alors comme:

\begin{equation}
    U_{P} = \sum_{j \in G} \ket{j}\bra{j} + \sum_{k} e^{-i \gamma \varepsilon_{k}} \sum_{j \in E^{(k)}} \ket{j}\bra{j} \,,
\end{equation} 

où $G$ et $E^{(k)}$ sont respectivement les variétés des états fondamentaux et états excités, et donc l'ensemble des solutions et des non-solutions de $\varphi$. Pour une espace de solutions non contraint, l'opérateur $U_{D}^{GM}$ est donné par:

\begin{equation}
    U_{D}^{GM} = \mathds{1} - (1 - e^{-i\beta}) (\ket{+}\!\bra{+}^{\otimes n}) \,.
\end{equation}

Les états fondamentaux et excités s'écrivent comme:
\begin{align}
    \ket{g} = \frac{1}{\sqrt{ \lvert G \rvert }} \sum_{j \in G} \ket{j}, \\
    \ket{e^{(k)}} = \frac{1}{\sqrt{ \lvert E^{(k)} \rvert }} \sum_{j \in E^{(k)}} \ket{j} \,.
\end{align}
Montrons par induction que l'état final suit la formule récursive~\ref{eq:recursive-formula} en évoluant l'état initial $\ket{\psi_{0}} = \ket{+}^{\otimes n}$ avec un circuit optimisé GM-QAOA de $p$ couches. 

\textbf{Cas 1 (1 couche):}

Calculons $\ket{\psi_{1}} = U_{P}(\gamma_{1}) \ket{\psi_{0}}$:
\begin{equation}
\begin{aligned}
    \ket{\psi_{1}} & = \left(\sum_{j \in G} \ket{j}\bra{j} + \sum_{k} e^{-i \gamma_{1} \varepsilon_{k}} \sum_{j \in E^{(k)}} \ket{j}\bra{j} \right) \left( \frac{1}{ \sqrt{ N }} \sum_{j'=0}^{N-1} \ket{j'} \right) \\
    & = \frac{1}{ \sqrt{ N }} \left( \sum_{j \in G} \ket{j} + \sum_{k} e^{-i \gamma_{1} \varepsilon_{k}} \sum_{j \in E^{(k)}} \ket{j} \right) \\
    & = \frac{1}{ \sqrt{ N }} \left( \sqrt{ \lvert G \rvert  } \ket{g} + \sum_{k} \sqrt{ \lvert E^{(k)} \rvert  } e^{-i \gamma_{1} \varepsilon_{k}} \ket{e^{(k)}} \right) \,,
\end{aligned}
\end{equation}
où la relation $\sum_{j \in G}\ket{j}\bra{j} \sum_{j'=0}^{N} \ket{j'} = \sum_{j \in G} \ket{j}$ a été utilisée. Calculons désormais $\ket{\psi_{2}} = U_{D} (\beta_{1})\ket{\psi_{1}}$:
\begin{equation}
\begin{aligned}
    \ket{\psi_{2}} &= \left( \mathbb{1} - \frac{1}{N} (1 - e^{-i\beta_{1}}) \sum_{i',j'=0}^{N} \ket{i'}\!\bra{j'} \right) \left( \frac{1}{ \sqrt{ N }} \left( \sqrt{ \lvert G \rvert  } \ket{g} + \sum_{k} \sqrt{ \lvert E^{(k)} \rvert  } e^{-i \gamma_{1} \varepsilon_{k}}  \ket{e^{(k)}} \right) \right) \\
    &=  \frac{1}{ \sqrt{ N }} \left( \sqrt{ \lvert G \rvert  } \ket{g} + \sum_{k} \sqrt{ \lvert E^{(k)} \rvert  } e^{-i \gamma_{1} \varepsilon_{k}}  \ket{e^{(k)}} \right. \\
    & \left. - \frac{1}{N} (1 - e^{-i\beta_{1}}) \left( \lvert G \rvert + \sum_{k} \lvert E^{(k)} \rvert e^{-i \gamma_{1} \varepsilon_{k}} \right) \sum_{i'=0}^{N} \ket{i'} \right) \\
    &=  \frac{1}{ \sqrt{ N }} \left( \sqrt{ \lvert G \rvert  } \ket{g} + \sum_{k} \sqrt{ \lvert E^{(k)} \rvert  } e^{-i \gamma_{1} \varepsilon_{k}}  \ket{e^{(k)}} \right. \\
    & \left. - \frac{1}{N} (1 - e^{-i\beta_{1}}) \left( \lvert G \rvert + \sum_{k} \lvert E^{(k)} \rvert e^{-i \gamma_{1} \varepsilon_{k}}   \right) \left( \left( \sqrt{ \lvert G \rvert  } \ket{g} + \sum_{k} \sqrt{ \lvert E^{(k)} \rvert } \ket{e^{(k)}} \right) \right) \right) \\
    &= \frac{1}{\sqrt{ N }} \left( \sqrt{ \lvert G \rvert } F_{1} \ket{g} + \sum_{k} \sqrt{ \lvert E^{(k)} \rvert } \bar{F}^{(k)} \ket{e^{(k)}} \right) \,,
\end{aligned}
\end{equation}
où 
\begin{align}
    F_{1} &= 1 - \frac{1}{N} (1 - e^{-i\beta_{1}}) \left(  \lvert G \rvert + \sum_{k'} \lvert E^{(k')} \rvert e^{-i \gamma_{1} \varepsilon_{k'}} \right) \\
    \bar{F}^{(k)}_{1} &= e^{-i\gamma_{1} \varepsilon_{k}} - \frac{1}{N} (1 - e^{-i\beta_{1}}) \left(  \lvert G \rvert + \sum_{k'} \lvert E^{(k')} \rvert e^{-i \gamma_{1} \varepsilon_{k'}} \right) \,.
\end{align}
Le cas initial a donc été prouvé.

\textbf{Cas 2 ($p$ couches):}
Supposons que $\ket{\psi} = U_{D}(\beta_{p}) U_{P}(\gamma_{p}) \dots U_{D}(\beta_{1})U_{P}(\gamma_{1})\ket{\psi_{0}}$ donne lieu à la relation suivante:
\begin{equation}
\begin{aligned}
    \ket{\psi} &= \frac{1}{\sqrt{ N }} \left( \sqrt{ \lvert G \rvert } F_{p} \ket{g_{i}} + \sum_{k} \sqrt{ \lvert E^{(k)} \rvert } \bar{F}^{(k)}_{p} \ket{e^{(k)}} \right)
\end{aligned}
\end{equation}
où
\begin{equation}
\begin{aligned}
    F_{p} &= F_{p-1} - \frac{1}{N} (1-e^{-i\beta_{p}}) \left( \lvert G \rvert   F_{p-1} + \sum_{k'} \lvert E^{(k')} \rvert \bar{F}^{(k')}_{p-1} e^{-i\gamma_{p}\varepsilon_{k'}} \right) \\
    \bar{F}^{(k)}_{p} &= e^{-i\gamma_{p} \varepsilon_{k}}\bar{F}_{p-1}^{(k)} - \frac{1}{N} (1-e^{-i\beta_{p}}) \left( \lvert G \rvert   F_{p-1} + \sum_{k'} \lvert E^{(k')} \rvert \bar{F}^{(k')}_{p-1} e^{-i\gamma_{p}\varepsilon_{k'}} \right)
\end{aligned}
\end{equation}

\newpage
\textbf{Cas 3 ($p+1$ couches):}
En utilisant la supposition précédente, appliquons $U_{P}$ en calculant $\ket{\psi_{1}} = U_{P} \ket{\psi_{0}}$:
\begin{equation}
\begin{aligned}
\ket{\psi_{1}} & = \left(\sum_{j \in G} \ket{j}\bra{j} + \sum_{k} e^{-i \gamma_{p+1} \varepsilon_{k}} \sum_{j \in E^{(k)}} \ket{j}\bra{j} \right) \left( \frac{1}{\sqrt{ N }} \left( \sqrt{ \lvert G \rvert } F_{p} \ket{g} + \sum_{k} \sqrt{ \lvert E^{(k)} \rvert } \bar{F}^{(k)}_{p} \ket{e^{(k)}} \right) \right) \\
& = \frac{1}{\sqrt{ N }} \left( \sum_{j \in G_{i}} \ket{j}\bra{j} + \sum_{i=0}^{Q-1} \sum_{k} e^{-i \gamma_{p+1} \varepsilon_{k}} \sum_{j \in E^{(k)}_{i}} \ket{j}\bra{j} \right) \left( F_{p} \sum_{j \in G} \ket{j} + \sum_{k} \bar{F}^{(k)}_{p} \sum_{j \in E^{(k)}}\ket{j} \right) \\
& = \frac{1}{\sqrt{ N }} \left( F_{p} \sum_{j \in G}\ket{j}  + \sum_{k} \bar{F}^{(k)}_{p} e^{-i \gamma_{p+1} \varepsilon_{k}} \sum_{j \in E^{(k)}} \ket{j} \right) \\
& = \frac{1}{\sqrt{ N }} \left(  F_{p} \sum_{j \in G} \ket{j} + \sum_{k} \bar{F}^{(k)}_{p} e^{-i \gamma_{p+1} \varepsilon_{k}} \sum_{j \in E^{(k)}} \ket{j} \right)  \\
& = \frac{1}{\sqrt{ N }} \left( \sqrt{ \lvert G \rvert  } F_{p} \ket{g} + \sum_{k} \sqrt{ \lvert E^{(k)} \rvert  }\bar{F}^{(k)}_{p} e^{-i \gamma_{p+1} \varepsilon_{k}} \ket{e^{(k)}} \right) \\
\end{aligned}
\end{equation}
Appliquons ensuite $U_{D}$ en calculant $\ket{\psi_{2}}=U_{D}\ket{\psi_{1}}$:
\begin{equation}
\begin{aligned}
\ket{\psi_{2}} &= \left( \mathbb{1} - \frac{1}{N} (1 - e^{-i\beta_{p+1}}) \sum_{i',j'=0}^{N} \ket{i'}\!\bra{j'} \right) \left( \frac{1}{\sqrt{ N }} \left( \sqrt{ \lvert G \rvert  } F_{p} \ket{g} + \sum_{k} \sqrt{ \lvert E^{(k)} \rvert  }\bar{F}^{(k)}_{p} e^{-i \gamma_{p+1} \varepsilon_{k}} \ket{e^{(k)}} \right) \right) \\
&= \frac{1}{\sqrt{ N }} \left( \left( \sqrt{ \lvert G \rvert  } F_{p} \ket{g} + \sum_{k} \sqrt{ \lvert E^{(k)} \rvert  }\bar{F}^{(k)}_{p} e^{-i \gamma_{p+1} \varepsilon_{k}} \ket{e^{(k)}} \right) \right. \\
& \left. - \frac{1}{N} (1 - e^{-i\beta_{p+1}}) \left(  \lvert G \rvert F_{p} + \sum_{k} \lvert E^{(k)} \rvert \bar{F}^{(k)}_{p} e^{-i \gamma_{p+1} \varepsilon_{k}}\right) \sum_{i'=0}^{N} \ket{i'} \right) \\
&= \frac{1}{\sqrt{ N }} \left( \left( \sqrt{ \lvert G \rvert  } F_{p} \ket{g} + \sum_{k} \sqrt{ \lvert E^{(k)} \rvert  }\bar{F}^{(k)}_{p} e^{-i \gamma_{p+1} \varepsilon_{k}} \ket{e^{(k)}} \right) \right. \\
& \left.- \frac{1}{N} (1 - e^{-i\beta_{p+1}}) \left(  \lvert G \rvert F_{p} + \sum_{k} \lvert E^{(k)} \rvert \bar{F}^{(k)}_{p} e^{-i \gamma_{p+1} \varepsilon_{k}}\right) \sum_{i'=0}^{N} \ket{i'} \right) \\
&= \frac{1}{\sqrt{ N }} \left( \sqrt{ \lvert G \rvert  } F_{p+1} \ket{g_{i}} + \sum_{k} \sqrt{ \lvert E^{(k)} \rvert  } \bar{F}^{(k)}_{p+1} e^{-i \gamma_{p+1} \varepsilon_{k}} \ket{e^{(k)}_{i}} \right) \\
\end{aligned}
\end{equation}
où
\begin{equation}
\begin{aligned}
F_{p+1} &= F_{p} - \frac{1}{N} (1-e^{-i\beta_{p+1}}) \left( \lvert G \rvert   F_{p} + \sum_{k'} \lvert E^{(k')} \rvert \bar{F}^{(k')}_{p} e^{-i\gamma_{p+1}\varepsilon_{k'}} \right) \\
\bar{F}^{(k)}_{p+1} &= e^{-i\gamma_{p+1} \varepsilon_{k}}\bar{F}_{p}^{(k)} - \frac{1}{N} (1-e^{-i\beta_{p+1}}) \left( \lvert G \rvert   F_{p} + \sum_{k'} \lvert E^{(k')} \rvert \bar{F}^{(k')}_{p} e^{-i\gamma_{p+1}\varepsilon_{k'}} \right)
\end{aligned}
\end{equation}

La formule récursive a donc été prouvée par induction.
\end{proof}

%-----------------------------------------------------------------------------%
