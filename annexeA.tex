\begin{comment}
\end{comment}


\chapter{Auto-réductibilité du circuit GM-QAOA}

%-----------------------------------------------------------------------------%

\begin{maintheorem}{Auto-réductibilité de GM-QAOA}{auto-reductibilite-gm-qaoa}
    Suppose that a Grover-Mixer quantum alternating operator ansatz (GM-QAOA) circuit efficiently generates approximate solutions to a self-reducible problem $\varphi$ in $\#P$. Then, this circuit can be modified, without altering the circuit's parameters, to approximately solve every sub-problem $\varphi_{c}$ of $\varphi$ while conserving the properties of GM-QAOA. 
\end{maintheorem}

Let $\varphi$ in $\#P$ be the problem instance of size $n$ we want to solve  with GM-QAOA. The instance can be encoded into a problem Hamiltonian $H_{p}$ with eigenvalues $E=0$ for solutions and $E=\varepsilon_{k}>0$ for non-solutions. The operator $U_{P}$ corresponding with $H_{p}$ can then be written as:
\begin{equation}
    U_{P} = \sum_{j \in G} \ket{j}\bra{j} + \sum_{k} e^{-i \gamma \varepsilon_{k}} \sum_{j \in E^{(k)}} \ket{j}\bra{j} \,,
\end{equation} 
with $G$ and $E^{(k)}$ respectively being the ground state and excited state manifolds, and thus the set of solutions and non-solutions to $\varphi$. For an unconstrained solution space, the operator $U_{D}$ is given by
\begin{equation}
    U_{D}^{GM-QAOA} = \mathds{1} - (1 - e^{-i\beta}) (\ket{+}\!\bra{+}^{\otimes n}) \,.
\end{equation}
Let $\varphi_{c}$ be the sub-problem of $\varphi$ with the last $q$ variables fixed to the bitstring $c$. We define $G_{c}$ and $E_{c}^{(k)}$ as the sets of solutions and non-solutions to $\varphi_{c}$. The ground and excited states are written as:
\begin{align}
    \ket{g_{c}} = \frac{1}{\sqrt{ \lvert G_{c} \rvert }} \sum_{j \in G_{c}} \ket{j}, \\
    \ket{e^{(k)}_{c}} = \frac{1}{\sqrt{ \lvert E^{(k)}_{c} \rvert }} \sum_{j \in E^{(k)}_{c}} \ket{j} \,.
\end{align}
Keeping track of the last $q$ qubits, we evolve the initial state $\ket{+}^{\otimes n}$ with an optimized GM-QAOA circuit of $p$ layers. We find, by induction, that the final state follows the recursive formula

\begin{equation}
    \ket{\psi_{\varphi}} = \frac{1}{\sqrt{ N }} \sum_{i=0}^{Q-1} \left( \sqrt{ \lvert G_{i} \rvert } F_{p} \ket{g_{i}} + \sum_{k} \sqrt{ \lvert E^{(k)}_{i} \rvert } \bar{F}^{(k)}_{p} \ket{e^{(k)}_{i}} \right) \ket{i} \,,
\end{equation}
\begin{align}
    F_{p} &= F_{p-1} - \frac{1}{N} (1-e^{-i\beta_{p}}) \left( \lvert G \rvert   F_{p-1} + \sum_{k'} \lvert E^{(k')} \rvert \bar{F}^{(k')}_{p-1} e^{-i\gamma_{p}\varepsilon_{k'}} \right) \,, \\
    \bar{F}^{(k)}_{p} &= e^{-i\gamma_{p} \varepsilon_{k}}\bar{F}_{p-1}^{(k)} - \frac{1}{N} (1-e^{-i\beta_{p}}) \left( \lvert G \rvert   F_{p-1} + \sum_{k'} \lvert E^{(k')} \rvert \bar{F}^{(k')}_{p-1} e^{-i\gamma_{p}\varepsilon_{k'}} \right) \,,
\end{align}

where $N=2^{n}$, $Q=2^{q}$ and $F_{0}=\bar{F}_{0}^{(k)}=1$. To solve the sub-problem $\varphi_{c}$, we make some modifications to the circuit. For each of the last $q$ qubits:
\begin{enumerate}
    \item Replace the initial $H$ gate by an $X$ gate conditioned on the value to which the qubit is being fixed.
    \item Remove the mixer Hamiltonian affecting the qubit.
\end{enumerate}
We find, by induction, that the final state after these modifications follows the same recursive formula with the following transformations:
\begin{align*}
    \sum_{i=0}^{Q-1} \ket{i} &\to \ket{c} \,, \\
    N &\to N_{Q} \,, \\
    G &\to G_{c} \,, \\
    E^{(k)} &\to E^{(k)}_{c} \,,
\end{align*}
where $N_{Q}=2^{n-q}$. The final state therefore conserves the same properties as the initial GM-QAOA circuit. We now suppose that the problem $\varphi$ is mapped to a two-level system, meaning that $\varepsilon_{k}=\varepsilon$ and therefore that $\lvert E_{c} \rvert = N_{Q} - \lvert G_{c} \rvert$. Using $\lvert G_{c} \rvert / N_{Q} \to 0$, we find that $F_{p,c}$ and $\bar{F}_{p,c}$ become:
\begin{equation}
\begin{aligned}
\tilde{F}_{p,c} &= \tilde{F}_{p-1,c} \\
&- \frac{1}{N_{Q}} (1-e^{-i\beta_{p}}) \left( N_{Q} \tilde{\bar{F}}_{p-1,c} e^{-i\gamma_{p}\varepsilon} \right) \,, \\
\tilde{\bar{F}}_{p, c} &= e^{-i\gamma_{p} \varepsilon_{k}} \tilde{\bar{F}}_{p-1, c} \\ 
&- \frac{1}{N_Q} (1-e^{-i\beta_{p}}) \left( N_{Q} \tilde{\bar{F}}_{p-1, c} e^{-i\gamma_{p}\varepsilon} \right) \,.
\end{aligned}
\end{equation}
Note that $\tilde{F}_{p,c}$ does not depend anymore on $\lvert G_{c} \rvert$. The success rate is then given by:
\begin{equation}
r_{\varphi_{c}} =  \left\lvert  \sqrt{ \frac{\lvert G_{c} \rvert }{N_{Q}} } \tilde{F}_{p, c} \right\rvert^{2} \,.
\end{equation}
This probability only changes with $\lvert G_{c} \rvert / N_{Q}$. In the JVV algorithm, the most probable variable is always picked as we follow the most probable path. For GM-QAOA, we are ensured that we pick the most probable variable after sampling a polynomial number of solutions, because every solution has the same amplitude. Suppose that at the step $q$ in the JVV procedure, we have fixed the last $q$ qubits to a bitstring $c$. At the following step $q'=q+1$, say we choose the bitstring $c'$ over the bitstring $\bar{c}'$. We then have that $\lvert G_{c'} \rvert \geq \lvert G_{\bar{c}'} \rvert$. Since $N_{Q} = \frac{N}{Q} = 2^{n-q}$, we find that
\begin{equation}
\frac{\lvert G_{c} \rvert }{N_{Q}} = \frac{\lvert G_{c'} \rvert + \lvert G_{\bar{c}'} \rvert }{2N_{Q'} } \leq \frac{\lvert G_{c'} \rvert }{N_{Q'}} \,.
\end{equation}
Therefore, we have that $\lvert G \rvert / N \leq \lvert G_{c} \rvert / N_{Q}$, and then that $r_{\varphi} \leq r_{\varphi_{c}}$.

%-----------------------------------------------------------------------------%
