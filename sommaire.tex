\begin{comment}
\end{comment}

En exploitant à la fois les ressources classiques et quantiques, les algorithmes variationnels quantiques peuvent exécuter des tâches computationnelles complexes sur des ordinateurs quantiques bruités, sans nécessiter de correction d'erreurs. Ces algorithmes visent notamment à résoudre des problèmes d'optimisation combinatoire, se plaçant ainsi en concurrence avec des algorithmes classiques perfectionnés au fil des décennies. Une approche complémentaire consiste à se tourner vers des problèmes intrinsèquement plus complexes: les problèmes de comptage. Ces derniers, qui consistent à déterminer le nombre de solutions aux problèmes de décision, pourraient se révéler plus accessibles aux algorithmes variationnels quantiques qu'aux algorithmes classiques. 

Dans le cadre de ce mémoire, un algorithme surnommé VQCount est introduit pour la résolution de problèmes de comptage. VQCount s'appuie sur l'équivalence entre l'échantillonnage aléatoire et le comptage approximatif pour estimer le nombre de solutions à un facteur multiplicatif près, en exploitant l'ansatz quantique à opérateurs alternants comme générateur de solutions. L'équivalence entre l'échantillonnage quasi uniforme et le comptage approximatif randomisé, due au travail de Jerrum, Valiant Vazirani, permet 

\textcolor{mydarkred}{\textit{Resultat}}

À l'aide de simulations de réseaux de tenseurs, la performance de VQCount avec de circuits de faible profondeur est étudiée sur des instances synthétiques de deux problèmes \textsf{\#P}-difficile, positif \#1-in-3SAT et positif \#NAE3SAT. 

Les résultats obtenus soulignent le potentiel et les limitations des algorithmes variationnels quantiques pour le comptage approximatif. 

