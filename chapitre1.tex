\begin{comment}
Problème NP vs #P
Théorème de Toda
Description du paper JVV
Complexité exacte vs approximative
Countage exact vs approximatif
Borne sur le comptage ("a tighter bound for counting max-weight solutions to 2SAT instances" ou l'équivalent pour 3SAT)
\end{comment}

\chapter{Complexité du comptage}

\begin{comment}
    \subsection*{Plan}
    
    \begin{enumerate}
        \item Introduire les problèmes algorithmiques difficiles
        \item Décrire les applications de ces problèmes
        \item Expliquer les prochaines sections
    \end{enumerate}
\end{comment}




%-----------------------------------------------------------------------------%

\section{Classes de complexité}

\begin{comment}
\subsection*{Plan}

\begin{enumerate}
    \item Décrire le but des classes de complexité
    \item Expliquer les propriétés des classes de complexité et leurs relations
    \item Définir comment quantifier la complexité d'un problème (temps contre espace)
    \item Expliquer la notation de la complexité (O notation) et les machines de Turing
    \item Décrire la tour de complexité (hiérarchie polynomiale) 
    \item Comparer les classes importantes: P et NP et \#P
    \item Établir la conjecture P != NP
    \item Mentionner le théorème de Toda
\end{enumerate}

\subsection*{Références}

1. Moore, Cristopher, and Stephan Mertens, The Nature of Computation (Oxford, 2011; online edn, Oxford Academic, 17 Dec. 2013), https://doi.org/10.1093/acprof:oso/9780199233212.001.0001, accessed 19 July 2024.

2. Arora, S. and Barak, B. Computational Complexity: A Modern Approach. (Cambridge University Press, Cambridge, 2009). doi:10.1017/CBO9780511804090.
\end{comment}



\begin{definition}[Classe de complexité \textsf{NP}]
    Un ensemble $A$ est contenu dans $\textsf{NP}$ si et seulement s'il existe un ensemble calculable en temps polynomial $B$ et un polynôme $p$ tel que

    \begin{equation*}
        x \in A \iff \text{ il existe un } y \in \Sigma^{*} \text{ satisfaisant } |y| \leq p(|x|) \text{ et } \braket{x, y} \in  B
    \end{equation*}

    est vrai pour tout $x \in \Sigma^{*}$.
\end{definition}





%-----------------------------------------------------------------------------%

\section{Problème de satisfaisabilité booléenne}

\subsection*{Plan}

\begin{enumerate}
    \item Introduire SAT
    \item Énumérer certaines applications de ce problème
    \item Faire le lien entre le problème de décision SAT et le problème de comptage SAT
    \item Introduire NAE3SAT et 1in3SAT
    \item Énoncer la réduction entre NAE3SAT/1in3SAT et 3SAT
    \item Introduire la transition de phase critique de ces problèmes
    \item Expliquer pourquoi prendre la version positive de ces problèmes n'est pas un problème
\end{enumerate}

\subsection*{Références}

1. Moore, Cristopher, and Stephan Mertens, The Nature of Computation (Oxford, 2011; online edn, Oxford Academic, 17 Dec. 2013), https://doi.org/10.1093/acprof:oso/9780199233212.001.0001, accessed 19 July 2024.

2. Arora, S. and Barak, B. Computational Complexity: A Modern Approach. (Cambridge University Press, Cambridge, 2009). doi:10.1017/CBO9780511804090.

3. Achlioptas, D., Chtcherba, A., Istrate, G. and Moore, C. The phase transition in 1-in-k SAT and NAE 3-SAT. Proceedings of the Annual ACM-SIAM Symposium on Discrete Algorithms (2001) doi:10.1145/365411.365760.


%-----------------------------------------------------------------------------%

\section{Intractabilité et approximations}

\subsection*{Plan}

\begin{enumerate}
    \item Expliquer le concept d'intractabilité
    \item Montrer la difficulté de résoudre des problèmes computationnels de manière exacte
    \item Expliquer les advantages des méthodes approximatives (temps polynomial, applications réelles)
    \item Introduire rigoureusement le concept d'approximation
\end{enumerate}

\subsection*{Références}


%-----------------------------------------------------------------------------%

\section{Complexité et bornes sur le comptage}

\subsection*{Plan}

\begin{enumerate}
    \item Décrire les résultats actuels en terme de comptage exact et approximatif
    \item Énumérer les algorithmes et les solveurs modernes (DPLL, \textit{survey propagation}, \textit{belief propagation})
    \item Mentionner les meilleures bornes sur les problèmes de comptage
\end{enumerate}


\subsection*{Références}

1. Wahlström, M. A Tighter Bound for Counting Max-Weight Solutions to 2SAT Instances. in Parameterized and Exact Computation (eds. Grohe, M. and Niedermeier, R.) 202–213 (Springer, Berlin, Heidelberg, 2008). doi:10.1007/978-3-540-79723-419.

2. Sinclair, A. and Jerrum, M. Approximate counting, uniform generation and rapidly mixing Markov chains. Information and Computation 82, 93–133 (1989).


%-----------------------------------------------------------------------------%

\section{Transitions de phase}

\subsection*{Plan}

\begin{enumerate}
    \item Expliquer les différentes transitions de phase et leurs intuitions
    \item Décrire l'objectif des algorithmes classiques locaux et globaux, comme le "belief propagation" ou le "survey propagation"
    \item Expliquer brièvement où se situe VQCount par rapport à ça
\end{enumerate}

\subsection*{Références}

1. Watrous, J. Quantum Computational Complexity. Preprint at https://doi.org/10.48550/arXiv.0804.3401 (2008).

2. Mézard, M. and Montanari, A. Information, Physics, and Computation. (Oxford University Press, Oxford, New York, 2009).

3. Survey propagation: An algorithm for satisfiability - Braunstein - 2005 - Random Structures amp; Algorithms - Wiley Online Library. https://onlinelibrary.wiley.com/doi/abs/10.1002/rsa.20057.

