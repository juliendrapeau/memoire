\begin{comment}
Problème NP vs #P
Théorème de Toda
Description du paper JVV
Complexité exacte vs approximative
Countage exact vs approximatif
Borne sur le comptage ("a tighter bound for counting max-weight solutions to 2SAT instances" ou l'équivalent pour 3SAT)
\end{comment}

\chapter{Complexité du comptage}

\begin{comment}
    \subsection*{Plan}
    
    \begin{enumerate}
        \item Introduire les problèmes algorithmiques difficiles
        \item Décrire les applications de ces problèmes
        \item Expliquer les prochaines sections
        \item Expliquer pourquoi on ne parle pas des machines de Turing (thèse de Church-Turing)
    \end{enumerate}
\end{comment}

Il est souvent favorable de décrire la théorie de la complexité à l'aide du concept de \textit{machine de Turing}. ... . Ce mémoire fera abstraction de ....


%-----------------------------------------------------------------------------%

\section{Classes de complexité}

\begin{comment}
\subsection*{Plan}

\begin{enumerate}
    \item Définir comment quantifier la complexité d'un problème (temps contre espace)
    \item Décrire le but des classes de complexité
    \item Expliquer les propriétés des classes de complexité et leurs relations
    \item Expliquer la notation de la complexité (O notation) et les machines de Turing
    \item Décrire la tour de complexité (hiérarchie polynomiale) 
    \item Comparer les classes importantes: P et NP et \#P
    \item Établir la conjecture P != NP
    \item Mentionner le théorème de Toda
    \item Parler de la these de church-turing pour les ordis quantiques
\end{enumerate}

\subsection*{Références}

1. Moore, Cristopher, and Stephan Mertens, The Nature of Computation (Oxford, 2011; online edn, Oxford Academic, 17 Dec. 2013), https://doi.org/10.1093/acprof:oso/9780199233212.001.0001, accessed 19 July 2024.

2. Arora, S. and Barak, B. Computational Complexity: A Modern Approach. (Cambridge University Press, Cambridge, 2009). doi:10.1017/CBO9780511804090.
\end{comment}

La théorie de la complexité s'intéresse à la classification des problèmes algorithmiques en \textit{classes de complexité}, c'est-à-dire en ensembles de problèmes de même complexité. Classifier un problème permet de caractériser les limites atteignables de sa résolution par un algorithme.

\textcolor{mydarkred}{\textit{Pourquoi est-ce qu'on veut faire ça?}}

Comment est-il possible de généraliser la complexité d'un problème? Pour ce faire, les classes de complexité se basent sur les ressources indispensables à la résolution du problème: le temps et la mémoire. Afin de trouver la solution à un problème, un programme doit effectuer un certain nombre d'opérations, limité dans le temps par le matériel informatique. On parle alors de \textit{complexité en temps}. Afin de produire un résultat final, le programme garde souvent en mémoire les résultats intermédiaires. Tous les résultats doivent être sauvegardé dans le matériel informatique afin de pouvoir être réutilisés ultérieurement. La quantité d'information conservée est aussi un facteur limitant pour le matériel informatique, on parle donc de \textit{complexité en espace}. 

Un problème de grande taille est plus 
\textcolor{mydarkred}{\textit{Parler de la dépendance en n.}}

Le temps et la mémoire quantifie bien les ressources nécessaires des algorithmes. Par contre, ceux-ci dépendent du matériel informatique utilisé. Il est attendu qu'un ordinateur des temps modernes soit bien plus performant qu'une des premières machines analogues. Comment retirer cette dépendence dans la notion de complexité? Pour ce faire, on fait appel à la notation asymptotique, communément appelé la notation $\mathcal{O}$. 

\textcolor{mydarkred}{\textit{Référence ici aux machines de Turing?}}


Ces ressources permettent la séparation de plusieurs problèmes: il est en effet souhaitable d'être capable de séparer les algorithmes efficaces de ceux qui ne le sont pas.

Les relations entre les classes de complexité...

Les problèmes de décision 

Qu'est-ce qui rend un algorithme efficace? La classe de complexité \textsf{P} tente de répondre à cette question.

Comment décrire l'efficacité d'un algorithme? Un algorithme est considéré comme efficace s'il résout un problème de la classe \textsf{P}.

\textcolor{mydarkred}{\textit{Rajouter des exemples!}}

\begin{maindefinition}{Classe de complexité \textsf{P}}{classe-p}
    Une fonction $A$ fait partie de la classe de complexité \textsf{P} si et seulement si un algorithme peut calculer $A(x)=y$ en temps polynomial, c’est-à-dire en temps $O(n^{c})$ pour une taille $n = \lvert x \rvert$ et une constante $c$.
\end{maindefinition}s

\begin{maindefinition}{Classe de complexité \textsf{NP}}{classe-np}
    Une fonction $A$ fait partie de la classe de complexité \textsf{NP} si et seulement si une fonction $B \in  \textsf{P}$ existe tel que
    \begin{equation*}
        A(x) = \exists y \mid B(x,y)
    \end{equation*}
    où $\lvert y \rvert = \mathrm{poly}(\lvert x \rvert)$.
\end{maindefinition}

On appelle $y$ le vérificateur (?).

\begin{maindefinition}{Classe de complexité \textsf{\#P}}{classe-sharp-p}
    Un fonction $A$ fait partie de la classe de complexité $\textsf{\#P}$ si et seulement si une fonction $B \in \textsf{P}$ existe tel que
    \begin{equation*}
        A(x) = \lvert \set{ y \mid B(x, y)} \rvert
    \end{equation*}
    où $\lvert y \rvert = \mathrm{poly}(\lvert x \rvert)$ pour toutes les valeurs $y$ prises par $B(x,y)$.
\end{maindefinition}

\begin{table}[h]
    \centering
    \begin{tabular}{c|c|c}
    Classes & Complexité & Description \\
    \hline
    ... & ... & ...
    \end{tabular}
    \caption{Sommaire des classes de complexité.}
    \label{tab:...}
\end{table}

\begin{subtheorem}{Théorème de Toda}{toda}
    ...
\end{subtheorem}


%-----------------------------------------------------------------------------%

\section{Problème de satisfaisabilité booléenne}

\subsection*{Plan}

\begin{comment}
\begin{enumerate}
    \item Introduire SAT
    \item Énumérer certaines applications de ce problème
    \item Faire le lien entre le problème de décision SAT et le problème de comptage SAT
    \item Introduire NAE3SAT et 1in3SAT
    \item Énoncer la réduction entre NAE3SAT/1in3SAT et 3SAT
    \item Introduire la transition de phase critique de ces problèmes
    \item Expliquer pourquoi prendre la version positive de ces problèmes n'est pas un problème
\end{enumerate}

\subsection*{Références}

1. Moore, Cristopher, and Stephan Mertens, The Nature of Computation (Oxford, 2011; online edn, Oxford Academic, 17 Dec. 2013), https://doi.org/10.1093/acprof:oso/9780199233212.001.0001, accessed 19 July 2024.

2. Arora, S. and Barak, B. Computational Complexity: A Modern Approach. (Cambridge University Press, Cambridge, 2009). doi:10.1017/CBO9780511804090.

3. Achlioptas, D., Chtcherba, A., Istrate, G. and Moore, C. The phase transition in 1-in-k SAT and NAE 3-SAT. Proceedings of the Annual ACM-SIAM Symposium on Discrete Algorithms (2001) doi:10.1145/365411.365760.
\end{comment}

Le problème de \textit{satisfaisabilité booléenne}, ou problème SAT est particulièrement important dans la théorie de la complexité. Montré comme \textsf{NP}-complet par le théorème de Cook-Levin~\cites{cookComplexityTheoremprovingProcedures1971}{levinUniversalSequentialSearch}, il fut à la base de la définition de \textsf{NP}-complétude et du problème $\textsf{P} = \textsf{NP}$. Celui-ci est aussi couramment utilisé dans la preuve de réductions de problèmes au sein de la classe de complexité \textsf
{NP}. 

\textcolor{mydarkred}{\textit{Rajouter des sources et expliquer plus en détails le théorème de Cook-Levin.}}

Le problème SAT a une multitude d'applications, comme…, en partie grâce à la facilité de formuler ces applications à l'aide de formules propositionnelles.

Une formule propositionnelle, ou une expression booléene, est un ensemble de variables booléenes, $x_{i} \in \set{ 0, 1 }$, reliées par des opérateurs booléeans de conjections ("ou", $\lor$), de disjonctions ("et", $\land$) ainsi que de négation ("non", $\neg$). Par exemple, l'expression $(x_{1} \land x_{2}) \lor \neg x_{3}$ est un exemple de formule booléenne.

Dans l'étude du problème SAT, les formules propositionnelles sont souvent exprimées en forme normale conjonctive (CNF). Celle-ci consiste en 


\begin{maindefinition}{Satisfaisabilité booléenne}{sat}
    Soit une constante $n \geq 1$ et une formule propositionnelle $\varphi(x_{1}, x_{2}, \dots, x_{n})$ où $x_{i} \in \set{ 0, 1 }$. Existe-il une assignation des variables $x_{1}, x_{2}, \dots, x_{n}$ telle que $\varphi$ est satisfaisable, c'est-à-dire que $\varphi(x_{1}, x_{2}, \dots, x_{n})=1$?
\end{maindefinition}

\begin{example}{Satisfaisabilité booléenne}{sat}
    L
\end{example}





%-----------------------------------------------------------------------------%

\section{Intractabilité et approximations}

\subsection*{Plan}

\begin{enumerate}
    \item Expliquer le concept d'intractabilité
    \item Montrer la difficulté de résoudre des problèmes computationnels de manière exacte
    \item Expliquer les advantages des méthodes approximatives (temps polynomial, applications réelles)
    \item Introduire rigoureusement le concept d'approximation
\end{enumerate}

\subsection*{Références}


%-----------------------------------------------------------------------------%

\section{Complexité et bornes sur le comptage}

\subsection*{Plan}

\begin{enumerate}
    \item Décrire les résultats actuels en terme de comptage exact et approximatif
    \item Énumérer les algorithmes et les solveurs modernes (DPLL, \textit{survey propagation}, \textit{belief propagation})
    \item Mentionner les meilleures bornes sur les problèmes de comptage
\end{enumerate}


\subsection*{Références}

1. Wahlström, M. A Tighter Bound for Counting Max-Weight Solutions to 2SAT Instances. in Parameterized and Exact Computation (eds. Grohe, M. and Niedermeier, R.) 202–213 (Springer, Berlin, Heidelberg, 2008). doi:10.1007/978-3-540-79723-419.

2. Sinclair, A. and Jerrum, M. Approximate counting, uniform generation and rapidly mixing Markov chains. Information and Computation 82, 93–133 (1989).


%-----------------------------------------------------------------------------%

\section{Transitions de phase}

\subsection*{Plan}

\begin{enumerate}
    \item Expliquer les différentes transitions de phase et leurs intuitions
    \item Décrire l'objectif des algorithmes classiques locaux et globaux, comme le "belief propagation" ou le "survey propagation"
    \item Expliquer brièvement où se situe VQCount par rapport à ça
\end{enumerate}

\subsection*{Références}

1. Watrous, J. Quantum Computational Complexity. Preprint at https://doi.org/10.48550/arXiv.0804.3401 (2008).

2. Mézard, M. and Montanari, A. Information, Physics, and Computation. (Oxford University Press, Oxford, New York, 2009).

3. Survey propagation: An algorithm for satisfiability - Braunstein - 2005 - Random Structures amp; Algorithms - Wiley Online Library. https://onlinelibrary.wiley.com/doi/abs/10.1002/rsa.20057.

