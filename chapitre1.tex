\begin{comment}
Problème NP vs #P
Théorème de Toda
Description du paper JVV
Complexité exacte vs approximative
Countage exact vs approximatif
Borne sur le comptage ("a tighter bound for counting max-weight solutions to 2SAT instances" ou l'équivalent pour 3SAT)
\end{comment}

\chapter{Complexité du comptage}

\begin{comment}
    \subsection*{Plan}
    
    \begin{enumerate}
        \item Introduire les problèmes algorithmiques difficiles
        \item Décrire les applications de ces problèmes
        \item Expliquer les prochaines sections
        \item Expliquer pourquoi on ne parle pas des machines de Turing (thèse de Church-Turing)
    \end{enumerate}
\end{comment}

Il est souvent favorable de décrire la théorie de la complexité à l'aide du concept de \textit{machine de Turing}. ... . Ce mémoire fera abstraction de ....


%-----------------------------------------------------------------------------%

\section{Classes de complexité}

\begin{comment}
\subsection*{Plan}

\begin{enumerate}
    \item Définir comment quantifier la complexité d'un problème (temps contre espace)
    \item Décrire le but des classes de complexité
    \item Expliquer les propriétés des classes de complexité et leurs relations
    \item Expliquer la notation de la complexité (O notation) et les machines de Turing
    \item Décrire la tour de complexité (hiérarchie polynomiale) 
    \item Comparer les classes importantes: P et NP et \#P
    \item Établir la conjecture P != NP
    \item Mentionner le théorème de Toda
    \item Parler de la these de church-turing pour les ordis quantiques
\end{enumerate}

\subsection*{Références}

1. Moore, Cristopher, and Stephan Mertens, The Nature of Computation (Oxford, 2011; online edn, Oxford Academic, 17 Dec. 2013), https://doi.org/10.1093/acprof:oso/9780199233212.001.0001, accessed 19 July 2024.

2. Arora, S. and Barak, B. Computational Complexity: A Modern Approach. (Cambridge University Press, Cambridge, 2009). doi:10.1017/CBO9780511804090.
\end{comment}

La théorie de la complexité s'intéresse à la classification des problèmes algorithmiques en \textit{classes de complexité}, c'est-à-dire en ensembles de problèmes de même complexité. Classifier un problème permet de caractériser les ressources nécessaires pour sa résolution par un algorithme. Les problèmes d'une même classe possède une difficulté inhérente similaire, ce qui permet le choix d'un algorithme et de ressources appropriées en conséquence. Savoir qu'un problème n'est pas réalistiquement résoluble, ou plus précisément intractable, limite les attentes. Un problème peut aussi être théoriquement résoluble sans que les algorithmes de pointe ne permettent sa résolution. Sachant ceci, la recherche dans cette direction peut s'avérer grandement utile. La théorie de la complexité cherche aussi à comparer les problèmes de différentes complexités. Ces comparaisons permettent de comprendre l'espace des problèmes en plus grande profondeur. Par exemple, il est évident que certains problèmes sont plus facile à résoudre que d'autres. Comparer des problèmes faciles avec des problèmes difficiles peut aider à comprendre ce qui rend un problème difficile et donc à trouver des algorithmes résolvant efficacement les problèmes plus complexes. Des liens, nommés réductions, peuvent aussi être définis au sein d'une même classe d'une complexité. Un algorithme efficace pour un problème pourrait aussi être efficace pour un problème similaire s'il existe une réduction entre ceux-ci. Les classes de complexité, de manière similaire au modèle de la machine de Turing, tentent de définir de manière abstraite la difficulté d'un problème. Peu importe le matériel informatique à notre disposition, un problème d'une classe donnée ne devrait pas changer de classe. Un problème trivial devrait rester trivial peu importe la quantité de ressources utilisée.  

Comment est-il possible de déterminer la complexité d'un problème? Pour ce faire, les classes de complexité se basent sur les ressources indispensables à la résolution du problème: le temps et la mémoire. Afin de trouver la solution à un problème, un programme doit effectuer un certain nombre d'opérations, limité dans le temps par le matériel informatique. On parle alors de \textit{complexité en temps}. Afin de produire un résultat final, le programme doit garder en mémoire les résultats intermédiaires. Ceux-ci doivent être sauvegardés dans le matériel informatique afin d'être réutilisés ultérieurement. Comme la quantité d'information conservée est aussi un facteur limitant pour le matériel informatique, on parle donc de \textit{complexité en espace}. 

La complexité en temps et en espace d'un problème est définie selon la taille de celui-ci. Un problème de plus grande taille est nécessairement plus complexe. \textcolor{mydarkred}{\textit{Pourquoi?}} Afin de capturer cette dépendence, on cherche à trouver une loi d'échelle encaspulant la difficulté d'un problème en fonction de sa taille. 

Le temps et la mémoire quantifient bien les ressources nécessaires des algorithmes. Par contre, ceux-ci dépendent du matériel informatique utilisé. Il est attendu qu'un ordinateur moderne soit bien plus performant qu'une des premières machines analogues. Comment retirer cette dépendence de la notion de complexité? Pour ce faire, on fait appel à la notation asymptotique, communément appelé la \textit{notation $\mathcal{O}$}. La notation asymptotique caractérise la vitesse de croissance d'une fonction en ne considérant que son comportement global à l'infini. Les coefficients ainsi que les termes asymptotiquement inférieurs ne sont pas considérés. Par exemple, pour une taille de problème $n$, la résolution de celui-ci pourrait demander un temps exponentiel $\mathcal{O}(2^{n})$ et une mémoire polynomiale $\mathcal{O}(n)$. On remarque qu'il n'y a aucune dépendence au matériel informatique: deux ordinateurs différents doivent effectuer le même nombre d'opération et sauvegarder la même quantité d'information. Un de ces ordinateurs pourraient toutefois résoudre le problème plus rapidement si celui-ci peut effectuer un plus grand nombre d'opérations par seconde ou accéder plus rapidement à sa mémoire. L'attrait des classes de complexité vient donc en partie de cette abstraction du matériel informatique.

La quantification de ces ressources permettent la séparation de plusieurs problèmes: il est en effet souhaitable d'être capable de séparer les algorithmes efficaces de ceux qui ne le sont pas. Commençons par définir deux classes de complexité particulièrement importantes: \textsf{P} et \textsf{NP}. Pour ce faire, il faut d'abord définir les \textit{problèmes de décision}. Ce type de problème regroupe simplement tous les problèmes pouvant se répondre par oui ou non. Pour tout problème de décision $A$, on peut représenter celui-ci par une fonction $A(x) \in \set{ 0, 1 }$. Les problèmes de décision se manifestent fréquemment, tant en informatique qu'en physique. Ceux-ci se présentent sous diverses formes: Est-ce qu'un nombre $x$ est premier? La configuration $x$ représente-elle un état fondamental du système donné? Est-ce qu'il existe un chemin $x$ parcourant une seule fois toutes les villes d'une région en parcourant au maximum une distance $d$?

% \begin{maindefinition}{Problème de décision}{probleme-decision}
%     Une fonction $A(x)$ est un problème de décision si
%     \begin{align*}
%         A(x) = 
%         \begin{cases}
%            0 \text{ si } x \text{ est une réponse «non» du problème} \\
%            1 \text{ si } x \text{ est une réponse «oui» du problème}
%         \end{cases}
%     \end{align*}
% \end{maindefinition}

Quand peut-on dire qu'un problème de décision est résoluble efficacement? La classe de complexité \textsf{P}, pour «temps polynomial», tente de répondre à cette question. Informellement, un problème de la classe \textsf{P} est un problème de décision qui peut être résolu en temps polynomial. Un problème est donc considéré comme efficacement résoluble, ou \textit{tractable}, s'il appartient à la classe \textsf{P}. 

\begin{maindefinition}{Classe de complexité \textsf{P}}{classe-p}
    Une fonction $A$ fait partie de la classe de complexité \textsf{P} si et seulement si un algorithme peut calculer 
    \begin{equation*}
        A(x)=\exists y
    \end{equation*}
    en temps polynomial, c’est-à-dire en temps $O(n^{c})$ pour une taille $n = \lvert x \rvert$ et une constante $c$, où $\lvert y \rvert = \mathrm{poly}(\lvert x \rvert )$.
\end{maindefinition}

Soit, par exemple, le problème de décision du test de primalité $A$. Ce problème cherche à déterminer si un entier naturel $x$ est premier ou composé. Ce problème peut être résolu, c'est-à-dire qu'il est possible de calculer $A(x)=\exists y$, en temps polynomial $\tilde{O}(\log(n)^{12})$~\cite{PRIMESAnnalsMathematics}, où la notation $\tilde{O}$ signifie que les termes poly-logarithmiques sont aussi cachés. \textcolor{mydarkred}{\textit{Rajouter des exemples!}}

La relation entre un calcul en temps polynomial et un calcul efficace semble évidente à prime abord, comme indiqué par la thèse de Cobham–Edmonds (\textcolor{mydarkred}{\textit{Citation!}}). Cependant, certains problèmes ne possèdent pas de solutions efficaces en pratique. Par exemple, un problème peut appartenir à la classe \textsf{P}, mais être doté d'un grand coefficient limitant le calcul. Cela n'étant pas le cas pour la majorité des problèmes, cette supposition s'avère malgré tout une bonne règle empirique.

Une deuxième classe particulièrement importante en théorie de la complexité est la classe \textsf{NP}, pour «temps polynomial non-déterministe». Celle-ci regroupe les problèmes de décision dont les solutions sont vérifiables en temps polynomial. Généralement, ces problèmes sont formulés sous la forme suivante: Existe-t-il une solution, vérifiable en temps polynomial, au problème donné. Une métaphore souvent utilisée pour la description d'un problème \textsf{NP} est celle d'une aiguille dans une botte de foin. Trouver cette aiguille parmi la quantité énorme de brins de foin est un défi de taille. Par contre, une fois l'aiguille trouvée, il n'y a aucun doute qu'il s'agit bien d'une aiguille. \textcolor{mydarkred}{\textit{Parler de non-déterministique. Est-ce pertinent comme il faut introduire les machines de Turing non-déterministe? Autrement, un problème appartient à la classe \textsf{NP} s'il est résoluble en temps polynomial par une machine de Turing non-déterministe.}}

\begin{maindefinition}{Classe de complexité \textsf{NP}}{classe-np}
    Une fonction $A$ fait partie de la classe de complexité \textsf{NP} si et seulement si une fonction $B \in  \textsf{P}$ existe tel que
    \begin{equation*}
        A(x) = \exists y \mid B(x,y)
    \end{equation*}
    où $\lvert y \rvert = \mathrm{poly}(\lvert x \rvert)$.
\end{maindefinition}

On appelle $B$ le vérificateur du problème de décision $A$ et $y$ le certificat ou le témoin pour l'entrée $x$. Un exemple de problème \textsf{NP} est le problème du commis voyageur. Soit un graphe $x$ représentant les villes d'un région particulière. La fonction $A$ détermine si ... \textcolor{mydarkred}{\textit{Continuer.}}

\textcolor{mydarkred}{\textit{Rajouter des exemples!}}

\textcolor{mydarkred}{\textit{On a donc que $P \subseteq NP$.}}
\textcolor{mydarkred}{\textit{NP pour temps non-déterministique.}}

\textcolor{mydarkred}{\textit{La conjecture "Exponential Time Hypothesis" suggère que certains problèmes dans NP prennent un temps exponentiel. \cite{impagliazzoComplexityKSAT2001} (voir https://arxiv.org/pdf/1611.04471 pour plus d'information)}}

\begin{maindefinition}{Classe de complexité \textsf{\#P}}{classe-sharp-p}
    Un fonction $A$ fait partie de la classe de complexité $\textsf{\#P}$ si et seulement si une fonction $B \in \textsf{P}$ existe tel que
    \begin{equation*}
        A(x) = \lvert \set{ y \mid B(x, y)} \rvert
    \end{equation*}
    où $\lvert y \rvert = \mathrm{poly}(\lvert x \rvert)$ pour toutes les valeurs $y$ prises par $B(x,y)$.
\end{maindefinition}


\begin{table}[h]
    \centering
    \begin{tabular}{c|c|c}
    Classes & Complexité & Description \\
    \hline
    ... & ... & ...
    \end{tabular}
    \caption{Sommaire des classes de complexité.}
    \label{tab:...}
\end{table}

\begin{subtheorem}{Théorème de Toda}{toda}
    ...
\end{subtheorem}

\textcolor{mydarkred}{\textit{Définir les problèmes de décision et de comptage ainsi que les classes de complexité de manière plus formelle (alphabet, )? Changer $B(x,y)$ par $xBy$?}}

\textcolor{mydarkred}{\textit{Expliquer plus le théorème de Cook-Levin.}}

\textcolor{mydarkred}{\textit{Discuter des relations entre les classes de complexité définies.}}

%-----------------------------------------------------------------------------%

\section{Problème de satisfaisabilité booléenne}

\begin{comment}
\subsection*{Plan}

\begin{enumerate}
    \item Introduire SAT
    \item Énumérer certaines applications de ce problème
    \item Faire le lien entre le problème de décision SAT et le problème de comptage SAT
    \item Introduire NAE3SAT et 1in3SAT
    \item Énoncer la réduction entre NAE3SAT/1in3SAT et 3SAT
    \item Introduire la transition de phase critique de ces problèmes
    \item Expliquer pourquoi prendre la version positive de ces problèmes n'est pas un problème
    \item Parler du comptage des problèmes SAT
    \item Introduire la transition de phase critique de ces problèmes
\end{enumerate}

\subsection*{Références}

1. Moore, Cristopher, and Stephan Mertens, The Nature of Computation (Oxford, 2011; online edn, Oxford Academic, 17 Dec. 2013), https://doi.org/10.1093/acprof:oso/9780199233212.001.0001, accessed 19 July 2024.

2. Arora, S. and Barak, B. Computational Complexity: A Modern Approach. (Cambridge University Press, Cambridge, 2009). doi:10.1017/CBO9780511804090.

3. Achlioptas, D., Chtcherba, A., Istrate, G. and Moore, C. The phase transition in 1-in-k SAT and NAE 3-SAT. Proceedings of the Annual ACM-SIAM Symposium on Discrete Algorithms (2001) doi:10.1145/365411.365760.
\end{comment}

Le problème de \textit{satisfaisabilité booléenne}, ou problème SAT est particulièrement important dans la théorie de la complexité. Montré comme \textsf{NP}-complet par le théorème de Cook-Levin~\cite{cookComplexityTheoremprovingProcedures1971,levinUniversalSequentialSearch}, il fut à la base de la définition de \textsf{NP}-complétude et du problème $\textsf{P} = \textsf{NP}$. Celui-ci est aussi couramment utilisé dans la preuve de réductions de problèmes au sein de la classe de complexité \textsf
{NP}. \textcolor{mydarkred}{\textit{Rajouter des sources.}}

Le problème SAT a une multitude d'applications, comme \textcolor{mydarkred}{\textit{Rajouter des exemples.}}, en partie grâce à la facilité de formuler ces applications à l'aide de formules propositionnelles.

Une formule propositionnelle, ou une expression booléene, est un ensemble de variables booléenes, $x_{i} \in \set{ 0, 1 }$, reliées par des opérateurs booléeans de conjections ("ou", $\lor$), de disjonctions ("et", $\land$) ainsi que de négation ("non", $\neg$). Par exemple, l'expression $(x_{1} \land x_{2}) \lor \neg x_{3}$ est une formule booléenne. Une litéral est définie comme une variable booléenes ou sa négation. 

Dans l'étude du problème SAT, les formules propositionnelles sont souvent exprimées en forme normale conjonctive (CNF). Celle-ci consiste en 


\begin{maindefinition}{Satisfaisabilité booléenne}{sat}
    Soit une constante $n \geq 1$ et une formule propositionnelle $\varphi(x_{1}, x_{2}, \dots, x_{n})$ où $x_{i} \in \set{ 0, 1 }$. Existe-il une assignation des variables $x_{1}, x_{2}, \dots, x_{n}$ telle que $\varphi$ est satisfaisable, c'est-à-dire que $\varphi(x_{1}, x_{2}, \dots, x_{n})=1$?
\end{maindefinition}

\begin{example}{Satisfaisabilité booléenne}{sat}
    L
\end{example}





%-----------------------------------------------------------------------------%

\section{Intractabilité et approximations}

\subsection*{Plan}

\begin{enumerate}
    \item Expliquer le concept d'intractabilité
    \item Montrer la difficulté de résoudre des problèmes computationnels de manière exacte
    \item Expliquer les advantages des méthodes approximatives (temps polynomial, applications réelles)
    \item Introduire rigoureusement le concept d'approximation
\end{enumerate}

\subsection*{Références}


%-----------------------------------------------------------------------------%

\section{Complexité et bornes sur le comptage}

\subsection*{Plan}

\begin{enumerate}
    \item Décrire les résultats actuels en terme de comptage exact et approximatif
    \item Énumérer les algorithmes et les solveurs modernes (DPLL, \textit{survey propagation}, \textit{belief propagation})
    \item Mentionner les meilleures bornes sur les problèmes de comptage
\end{enumerate}


\subsection*{Références}

1. Wahlström, M. A Tighter Bound for Counting Max-Weight Solutions to 2SAT Instances. in Parameterized and Exact Computation (eds. Grohe, M. and Niedermeier, R.) 202–213 (Springer, Berlin, Heidelberg, 2008). doi:10.1007/978-3-540-79723-419.

2. Sinclair, A. and Jerrum, M. Approximate counting, uniform generation and rapidly mixing Markov chains. Information and Computation 82, 93–133 (1989).


%-----------------------------------------------------------------------------%

\section{Transitions de phase}

\subsection*{Plan}

\begin{enumerate}
    \item Expliquer les différentes transitions de phase et leurs intuitions
    \item Décrire l'objectif des algorithmes classiques locaux et globaux, comme le "belief propagation" ou le "survey propagation"
    \item Expliquer brièvement où se situe VQCount par rapport à ça
\end{enumerate}

\subsection*{Références}

1. Watrous, J. Quantum Computational Complexity. Preprint at https://doi.org/10.48550/arXiv.0804.3401 (2008).

2. Mézard, M. and Montanari, A. Information, Physics, and Computation. (Oxford University Press, Oxford, New York, 2009).

3. Survey propagation: An algorithm for satisfiability - Braunstein - 2005 - Random Structures amp; Algorithms - Wiley Online Library. https://onlinelibrary.wiley.com/doi/abs/10.1002/rsa.20057.

