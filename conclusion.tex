\Conclusion % Chapitre qui ne sera pas numéroté si IntroConcluSansNombre est Vrai

%-----------------------------------------------------------------------------%

Ce travail introduit VQCount, un algorithme variationnel quantique basé sur l'algorithme de Jerrum, Valiant et Vazirani pour le comptage approximatif de modèles à partir de l'échantillonnage aléatoire de l'état préparé par l'algorithme quantique à opérateurs alternants. La validité de VQCount est établie dans une limite où la forme de l'algorithme de Grover est retrouvée. Comparativement aux travaux précédents, VQCount nécessite un nombre d'échantillons exponentiellement plus petit pour obtenir une estimation à un facteur multiplicatif près du compte exact lorsque le nombre de solutions est exponentiel. La performance de VQCount est étudiée numériquement comme méthode heuristique, en utilisant la contraction de réseaux de tenseurs pour la simulation des circuits quantiques composant l'algorithme. Pour les problèmes de dénombrement \#NAE3SAT et \#1-in-3SAT, un compromis est établi entre le taux de succès et l'uniformité de l'échantillonnage de solutions. De plus, un gain exponentiel du nombre d'échantillons nécessaire est observé en comparaison à l'échantillonnage par rejet naïf.  

L'algorithme VQCount n'introduit que la fondation d'un algorithme variationnel quantique pour le comptage. La seule technique de post-traitement utilisée est le rejet des échantillons n'appartenant pas à l'ensemble de solutions, mais des techniques plus élaborées, comme la mise à jour de clusters sans rejet à température nulle~\cite{ochoaFeedingMultitudePolynomialtime2019}, pourrait être employée. Les algorithmes de comptage approximatif basés sur l'échantillonnage aléatoire ont aussi grandement évolués depuis le travail de Jerrum, Valiant et Vazirani. Par exemple, l'algorithme introduit par ces derniers peut être modifié pour obtenir des garanties sur le compte approximatif obtenu même pour des distributions de solutions non-uniforme~\cite{gomesSamplingModelCounting2007}. Ces nouvelles techniques pourraient être employées pour rivaliser avec les algorithmes classiques modernes.


