\Conclusion % Chapitre qui ne sera pas numéroté si IntroConcluSansNombre est Vrai

%-----------------------------------------------------------------------------%

Ce travail introduit VQCount, un algorithme variationnel quantique basé sur l'algorithme de JVV pour le comptage approximatif de modèles à partir de l'échantillonnage aléatoire de l'état préparé par QAOA. La validité de VQCount est établie dans une limite où la forme de l'algorithme de Grover est retrouvée. Comparativement à des travaux précédents, VQCount nécessite un nombre d'échantillons exponentiellement plus petit pour obtenir une estimation à un facteur multiplicatif près du compte exact lorsque le nombre de solutions est exponentiel. La performance de VQCount est étudiée numériquement en tant que méthode heuristique, en utilisant la contraction de réseaux de tenseurs pour simuler les circuits quantiques composant l'algorithme. Pour les problèmes de dénombrement \#NAE3SAT et \#1-in-3SAT, un compromis est établi entre le taux de succès et l'uniformité de l'échantillonnage de solutions. 





L'algorithme VQCount n'introduit que la fondation d'un algorithme variationnel quantique pour le comptage. 

% \textcolor{mydarkred}{\textit{Vérifier le formatage de la bibliographie.}}

